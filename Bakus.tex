\documentclass{beamer}

\usepackage[utf8]{inputenc}
\usepackage[russian]{babel}
\usepackage[T2A]{fontenc}
\usepackage{graphics}
\usepackage{textpos}


\author[Алиев П.Н. Чирков А.В.]{Алиев П.Н. Чирков А.В. группа 2262}
\title{Джон Бэкус}
\subtitle{Создатель первого высокоуровнего языка программирования}
\date[2025г.]{25 февраля 2025г.}
\usetheme{Madrid}

\begin{document}

\AtBeginSection[]
{
  \begin{frame}
    \frametitle{Содержание}
    \tableofcontents[currentsection]
  \end{frame}
}

\begin{frame}
\titlepage
\end{frame}

\section{Биография}
\begin{frame}
	\frametitle{Студенческие годы}
	\begin{textblock*}{3cm}(9cm,-1.5cm)
	\includegraphics[width=3cm]{image/bakus.png}
	\end{textblock*}
	\begin{textblock*}{9cm}(0cm,-2.5cm)
	Джон Бэкус родился 3 декабря 1924 года в Филадельфии. Его отец был главным инженером-химиком в компании по производству нитроглицерина. 

	Джон закончил школу в Вашингтоне, затем осенью 1942 года поступил в Виргинский университет на химический факультет. Когда Бэкус поступил в 			университет, единственным занятием, которое он посещал раз в неделю, был урок музыки. К концу второго семестра, в 1943 году, руководство университета решило, что дальнейшее пребывание юноши в стенах учебного заведения излишне: Бэкус был отчислен.
	\end{textblock*}
\end{frame}

\begin{frame}
\frametitle{Поиск жизненного пути}
После отчисления молодой Джон Бэкус в чине капрала был принят в дружные ряды тихоокеанской ПВО. Однако в боевых действиях Бэкус участия не принимал.
После демобилизации Джон вернулся в США и поселился в Нью-Йорке. Перед ним вновь встал вопрос выбора профессии. Ничто, кроме музыки, его не привлекало. Поскольку ему сильно хотелось иметь хорошую аппаратуру для прослушивания музыки, он поступил в школу радиотехники: получив там необходимые навыки, он смог бы самостоятельно сконструировать музыкальную аппаратуру.
\end{frame}

\begin{frame}
\frametitle{Поиск жизненного пути}
Однажды один преподаватель попросил Джона Бэкуса помочь ему с построением графиков частотных характеристик усилителя. Вычисления были несложными, но их обилие утомляло. Неожиданно эти повторяющиеся математические операции заинтересовали Бэкуса. Преподаватель по ремонту теле- и радиоаппаратуры пробудил в нем интерес к математике и убедил его продолжить образование в Колумбийском университете

\begin{figure}
\includegraphics[width=5cm]{image/images.png}
\end{figure}
\end{frame}

\section{IBM}

\begin{frame}
\frametitle{Начало работы в IBM}

\begin{itemize}
	\item 1949г. - получил степень магистра математики в Колумбийском университете
	\item По приглашению Рекса Сибера — одного из изобретателей машины SSEC (Selective Sequence Electronic Calculator) – Бэкус поступил на работу программистом в фирму IBM
	\item Через год во главе небольшой группы программистов разработал интерпретатор Speedcoding для компьютера IBM 701.
\end{itemize}

\begin{block}{Примечание}
Selective Sequence Electronic Calculator (SSEC) был одной из первых разработок IBM в новой области электронных вычислительных устройств на вакуумных лампах. Этот, так сказать, компьютер не имел памяти, а весь ввод и вывод происходил посредством перфолент. SSEC морально устарел уже в 1952 году и был демонтирован
\end{block}
\end{frame}


\begin{frame}
\frametitle{Новый проект}
В 1954 году компания IBM запустила новый проект – IBM 704. В отличие от ламповой ЭВМ 701, новый проект был электронно-магнитным. IBM 704 предоставила программистам универсальный набор команд для работы, в том числе с числами с плавающей запятой. Реализовывать на языке ассемблера алгоритмы обработки чисел с плавающей запятой нелегко. А программировать в ту пору приходилось в основном только математические формулы, и никаких математических сопроцессоров не было. В конечном итоге Джон Бэкус стал все больше задумываться над тем, как создать язык, независимый от архитектуры машины и позволяющий легко программировать математические формулы.
\end{frame}

\section{Fortran}

\begin{frame}
\frametitle{Разработчики Fortran}
\begin{itemize}
	\item Роберт Нельсон (Robert Nelson) 
	\item Харлан Херрик (Harlan Herrick)
	\item Льюис Хэйт (Lois Haibt)
	\item Рой Нат (Roy Nutt)
	\item Ирвинг Циллер (Irving Ziller)
	\item Шелдон Бест (Sheldon Best)
	\item Дэвид Сэйр (David Sayre), 
	\item Ричард Голдберг (Richard Goldberg)
	\item Питер Шеридан (Peter Sheridan)
\end{itemize}

\end{frame}
 gdfbmdkfmbzkdfmk
\begin{frame}
\frametitle{Два часа на игры}
В своих воспоминаниях Бэкус пишет, что из-за свойственной ему лени он создал такую систему управления группой, что ему, собственно, и делать-то ничего не приходилось. Наибольшую сложность представляла для него только задача, как заставить членов группы не тратить столько времени на игры. Программисты из группы Бэкуса во время ланча любили сразиться в шахматы (правда, в то время еще не виртуальные, а настоящие). И никакие угрозы не помогали в борьбе с этим: раньше 14:00 они обычно игру не заканчивали. С тех пор некоторые думают, что «настоящие» программисты должны тратить не менее двух часов в день на игры.

\end{frame}

\begin{frame}
\frametitle{Основные понятия нового языка}
\begin{itemize}
	\item оператор присваивания
	\item индексируемые переменные
	\item оператор DO
	\item конструкции IF
	\item циклы
	\item упростилось программирование формул
\end{itemize}

\begin{exampleblock}{Пример}
формула $D=B^2-4AC$, программирование которой даже на современном языке ассемблера потребует десяток строк кода, на новом языке записывалась как $D=B**2-4*A*C$
\end{exampleblock}
\end{frame}

\begin{frame}
\frametitle{Работа над компилятором}
Работа над языком шла быстро, чего нельзя было сказать о разработке компилятора. Бекус понимал, что развеять сомнения в возможностях «автоматического» программирования, то есть написания программ на языках высокого уровня, нелегко. Программы на Fortran должны быть такими же быстродействующими и надежными, как и написанные в машинных кодах или на языке ассемблера.

Из трех лет, затраченных на разработку проекта в целом, более двух лет заняла работа над компилятором. Если первое сообщение о создании языка группа сделала в 1954 году, то о разработке компилятора — только в апреле 1957 года.
\end{frame}

\begin{frame}
\frametitle{Первая страница руководства по Fortran с автографом Бэкуса}
\centering
\includegraphics[width = 10cm]{image/firstPageFortran.png}

\end{frame}

\section{Другие творения Бэкуса}
\begin{frame}
\frametitle{Algol}
\centering
\includegraphics[width = 7cm]{image/BNF.png}
\end{frame}

\begin{frame}
\frametitle{FP}

\centering
\includegraphics[width = 10cm]{image/FP.png}
\end{frame}

\section{Заключение}
\begin{frame}
\frametitle{Память о Бэкусе}

\begin{textblock*}{4cm}(0cm,-2.5cm)
\centering
\includegraphics[width = 4cm]{image/deadBack.png}
\end{textblock*}

\begin{textblock*}{8cm}(4.2cm,-3cm)

Джон Бэкус ушел из жизни 17 марта 2007 года. Ему было 82. Несмотря на то, что в начале пути он не знал, что делать со своей жизнью, судьба приняла это решение за него.

Кроме премии Алана Тьюринга, о которой было сказано выше, Джон Бэкус в 1976 году был награжден Национальной медалью за вклад в науку.

«Иногда стимулом к изобретению становится не полет творческой мысли и не необходимость, а желание отдохнуть от нудной и тяжелой работы. Джон Бэкус убежден, что именно это заставило его разработать язык, в значительной мере автоматизирующий работу программиста», – пишет Лесли Гофф.
\end{textblock*}
\end{frame}

\end{document}
